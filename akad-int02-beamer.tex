%
% Erstellt von Daniel Falkner
% daniel.falkner@akad.de
% 
\documentclass[xcolor=dvipsnames]{beamer}
\usepackage[T1]{fontenc}
\usepackage[utf8]{inputenc}
\usepackage[ngerman]{isodate}
\usepackage[justification=centering,figurename=Abb.]{caption}
\usepackage{listings}
\usepackage{color}

\definecolor{mygreen}{rgb}{0,0.6,0}
\definecolor{mygray}{rgb}{0.5,0.5,0.5}
\definecolor{mymauve}{rgb}{0.58,0,0.82}

\lstset{ %
  backgroundcolor=\color{white},   % choose the background color; you must add \usepackage{color} or \usepackage{xcolor}
  basicstyle=\footnotesize,        % the size of the fonts that are used for the code
  breakatwhitespace=false,         % sets if automatic breaks should only happen at whitespace
  breaklines=true,                 % sets automatic line breaking
  captionpos=b,                    % sets the caption-position to bottom
  commentstyle=\color{mygreen},    % comment style
  deletekeywords={...},            % if you want to delete keywords from the given language
  escapeinside={\%*}{*)},          % if you want to add LaTeX within your code
  extendedchars=true,              % lets you use non-ASCII characters; for 8-bits encodings only, does not work with UTF-8
 % frame=single,                    % adds a frame around the code
  keepspaces=true,                 % keeps spaces in text, useful for keeping indentation of code (possibly needs columns=flexible)
  keywordstyle=\color{blue},       % keyword style
  language=Octave,                 % the language of the code
  morekeywords={*,...},            % if you want to add more keywords to the set
  numbers=left,                    % where to put the line-numbers; possible values are (none, left, right)
  numbersep=5pt,                   % how far the line-numbers are from the code
  numberstyle=\tiny\color{mygray}, % the style that is used for the line-numbers
  rulecolor=\color{black},         % if not set, the frame-color may be changed on line-breaks within not-black text (e.g. comments (green here))
  showspaces=true,                % show spaces everywhere adding particular underscores; it overrides 'showstringspaces'
  showstringspaces=true,          % underline spaces within strings only
  showtabs=true,                  % show tabs within strings adding particular underscores
  stepnumber=1,                    % the step between two line-numbers. If it's 1, each line will be numbered
  stringstyle=\color{mymauve},     % string literal style
  tabsize=2,                       % sets default tabsize to 2 spaces
  title=\lstname                   % show the filename of files included with \lstinputlisting; also try caption instead of title
}

\usetheme{Warsaw}
\usecolortheme[named=OliveGreen]{structure}
\renewcommand\thempfootnote{\arabic{mpfootnote}}

\newcommand*{\Title}{Präsentation der Website} %Titel
\subtitle{Modul INT02} %Untertitel
\newcommand*{\Author}{Daniel Falkner + Eugen Grinschuk} %Name
\institute{AKAD Pinneberg + Stuttgart} %Uni
\titlegraphic{\includegraphics[scale=0.2]{akad_logo.png}} %Logo

\title{\Title}
\author{\Author}
\date{\today}

%Pdf Metainformationen
\subject{\Title}
\keywords{}

\begin{document}

%Titelseite
\begin{frame}
    \titlepage
\end{frame}

%Logo auf allen weiteren Folien
%\logo{\includegraphics[scale=0.1]{akad_logo.png}}

%Inhaltsverzeichniss
\frame{\tableofcontents} 


\section{Über uns}
\begin{frame} %%Eine Folie
  \frametitle{Über uns} %%Folientitel
  \begin{block}{Wer sind wir?}
	  \begin{itemize}
  		\item Daniel Falkner
	  	\item Eugen Grinschuk
	  \end{itemize}
  \end{block}
\end{frame}

\begin{frame} %%Eine Folie
  \frametitle{Über uns} %%Folientitel
  \framesubtitle{Daniel Falkner} %%Fielenuntertitel
  \begin{block}{Daniel Falkner}
	  \begin{itemize}
  		\item T-Systems Telekom IT
	  \end{itemize}
  \end{block}
\end{frame}

\begin{frame} %%Eine Folie
  \frametitle{Über uns} %%Folientitel
  \framesubtitle{Eugen Grinschuk} %%Fielenuntertitel
  \begin{block}{Eugen Grinschuk}
	  \begin{itemize}
  		\item T-Systems
	  \end{itemize}
  \end{block}
\end{frame}


\section{Tools}
\begin{frame} %%Eine Folie
  \frametitle{Tools} %%Folientitel
  \framesubtitle{Was verwenden wir?} %%Fielenuntertitel
  \begin{block}{}
	  \begin{itemize}
		\item Cross Platform (Windows, Mac, Linux)
  		\item Texteditor
	  	\item Git
		\item Php
		\item Apache
		\item HTML
		\item CSS
		\item Javascript
	  \end{itemize}
  \end{block}
\end{frame}

\section{CodeDemo}
\begin{frame}[fragile]
\frametitle{Code Demo}

\begin{lstlisting}[language=HTML]
<h1>Willkommen</h1>
<p>auf den Seiten der Firma Mega Busreisen - Max Mustermann</p>

<img src="images/reisen.jpg" alt="Foto">

\end{lstlisting}
\end{frame}



\subsection*{Ende}
\begin{frame}
	\begin{block}{}	
		\begin{center}
			Vielen Dank für Ihre Aufmerksamkeit. \\
			\Author{}
		\end{center}	
	\end{block}
\end{frame}

\end{document}


